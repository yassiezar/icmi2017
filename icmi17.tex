\documentclass[format=sigconf, review=true, screen=true, anonymous=true]{acmart}

\begin{document}

\acmConference[ICMI'17]{International Conference on Multimodal Interaction}{November 2017}{Glascow, Scotland}

\title[SHORT TITLE]{LONG TITLE}
\subtitle{SUBTITLE}

\author{AUTHOR 1}
\orcid{wtf is orcid}
\affiliation{%
  \insitution{INSTITUTION}
  \department{DEPARTMENT}
  \streetaddress{ADDRESS}
  \city{CITY}
  \state{STATE}
  \postcode{POST CODE}
  \country{COUNTRY}}
\email{EMAIL}

\thanks{THANKS}

\terms{KEY TERMS}
\keywords{KEY WORDS}

\acmYear{2017}
\received{July 2017}

\maketitle

\begin{abstract}
  hello
\end{abstract}

\section{Introduction}

What is the problem? reference AAAI paper
What is the contribution of this paper?

The UK's Royal National Institute for the Blind (RNIB), a leading organisation in the area, has identified a number of challenges for the modern blind and visually impaired (henceforth reffered to as the VI) person. These include the latter's ability to safely and independently use public transport and navigate in unfamiliar environments~\cite{rnib-objectives}. Recent technological advances in the fields of mobile computing and computer vision have allowed for new and innovative solutions to come to the fore to address these challenges. 

To this end, we have proposed a mobile device-based navigation system that caters to the needs of the blind and visually impaired that is based on a Google Project Tango device {CITE AAAI PAPER}. A Tango-enabled device comes pre-equipped with powerful image-processing, localisation and depth-perception capabilities and is built on top of a standard Android platform, giving us access to the entire set of input/output options that Android has to offer. The proposed system will use multiple feedback modes to guide a user toward a target destination while providing information on any oncoming obstacles.

In this paper, we discuss in detail how one of these feedback modes are used in our system. We also discuss the tests we performed to determine how effective this mode is at directing a user to completing a given task, as well as how the mode's parameter values affect a user's performance at completing the aforementioned task. 

The remainder of this paper is organised as follows: HOW IS PAPER ORGANISED?

\section{Previous Work}

\subsection{System Description}

\section{Tests Performed}

\section{Results}

\section{Discussion}

\section{Conclusion}

\bibliographystyle{ACM-Reference-Format}
\bibliography{icmi17}

\end{document}
